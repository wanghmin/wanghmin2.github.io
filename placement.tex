\documentclass[acmtog,anonymous,review]{acmart}
\acmSubmissionID{papers\_150}

\usepackage{booktabs} % For formal tables
\usepackage{wrapfig}
\usepackage{makecell}
\usepackage{subfigure}

% TOG prefers author-name bib system with square brackets
\citestyle{acmauthoryear}
%\setcitestyle{nosort,square} % nosort to allow for manual chronological ordering



\usepackage[ruled]{algorithm2e} % For algorithms
\renewcommand{\algorithmcfname}{ALGORITHM}
\SetAlFnt{\small}
\SetAlCapFnt{\small}
\SetAlCapNameFnt{\small}
\SetAlCapHSkip{0pt}

% Metadata Information
\acmJournal{TOG}
%\acmVolume{38}
%\acmNumber{4}
%\acmArticle{39}
%\acmYear{2019}
%\acmMonth{7}

% Copyright
%\setcopyright{acmcopyright}
%\setcopyright{acmlicensed}
%\setcopyright{rightsretained}
%\setcopyright{usgov}
%\setcopyright{usgovmixed}
%\setcopyright{cagov}
%\setcopyright{cagovmixed}

% DOI
%\acmDOI{0000001.0000001_2}

% Paper history
%\received{February 2007}
%\received{March 2009}
%\received[final version]{June 2009}
%\received[accepted]{July 2009}


% Document starts
\begin{document}
% Title portion
\title{Automatic Digital Garment Initialization from Sewing Patterns}

% DO NOT ENTER AUTHOR INFORMATION FOR ANONYMOUS TECHNICAL PAPER SUBMISSIONS TO SIGGRAPH 2024!
%\author{Huamin Wang}
%\orcid{1234-5678-9012-3456}
%\affiliation{%
%  \institution{College of William and Mary}
%  \streetaddress{104 Jamestown Rd}
%  \city{Williamsburg}
%  \state{VA}
%  \postcode{23185}
%  \country{USA}}
%\email{gang_zhou@wm.edu}
%\author{Valerie B\'eranger}
%\affiliation{%
%  \institution{Inria Paris-Rocquencourt}
%  \city{Rocquencourt}
%  \country{France}
%}
%\email{beranger@inria.fr}
%\author{Aparna Patel}
%\affiliation{%
% \institution{Rajiv Gandhi University}
% \streetaddress{Rono-Hills}
% \city{Doimukh}
% \state{Arunachal Pradesh}
% \country{India}}
%\email{aprna_patel@rguhs.ac.in}
%\author{Huifen Chan}
%\affiliation{%
%  \institution{Tsinghua University}
%  \streetaddress{30 Shuangqing Rd}
%  \city{Haidian Qu}
%  \state{Beijing Shi}
%  \country{China}
%}
%\email{chan0345@tsinghua.edu.cn}
%\author{Ting Yan}
%\affiliation{%
%  \institution{Eaton Innovation Center}
%  \city{Prague}
%  \country{Czech Republic}}
%\email{yanting02@gmail.com}
%\author{Tian He}
%\affiliation{%
%  \institution{University of Virginia}
%  \department{School of Engineering}
%  \city{Charlottesville}
%  \state{VA}
%  \postcode{22903}
%  \country{USA}
%}

%\renewcommand\shortauthors{Zhou, G. et al}


%
% The code below should be generated by the tool at
% http://dl.acm.org/ccs.cfm
% Please copy and paste the code instead of the example below.
%
\begin{CCSXML}
	<ccs2012>
	<concept>
	<concept_id>10010147.10010371.10010352.10010379</concept_id>
	<concept_desc>Computing methodologies~Physical simulation</concept_desc>
	<concept_significance>500</concept_significance>
	</concept>
	</ccs2012>
\end{CCSXML}
\ccsdesc[500]{Computing methodologies~Physical simulation}


\keywords{Physics-based cloth simulation, numerical optimization, digital fashion, local minima, sewing pattern}



\maketitle
\section{Introduction}
With the rise of digital fashion businesses and the progress in generative AI models, digital sewing patterns have become more accessible and affordable. This advancement raises an intriguing question: how can we effortlessly convert digital sewing patterns into well-fitted digital garments on human avatars, all through a fully automated process? This capability is in high demand for a range of digital fashion and entertainment applications, as it forms a key component in the automated creation of 3D garments and characters.

Unfortunately, when presented with a sewing pattern and its sewing relationships, cloth simulation often grapples with non-uniqueness, resulting in visual artifacts primarily due to local minima. These artifacts can manifest as self-folding in Fig.~\ref{fig:sleeve}a, cloth-body intersection in Fig.~\ref{fig:sleeve}b, or misplaced pieces stuck outside of the body in Fig.~\ref{fig:sleeve}c, depending on how the simulation objective is defined and optimized. To mitigate the local minima issue, a natural solution is to employ a suitable initialization. In essence, the goal of an initialization is to position the sewing pieces around the human body without folding or intersection, thus enabling the generation of visually acceptable digital garments through simulation. An intersection-free initialization is also important to simulators that rely on continuous collision detection and interior point methods~\cite{Li:2020:IPC}.

\bibliographystyle{ACM-Reference-Format}
\bibliography{publications}



\end{document}
